\documentclass[12pt,a4paper,oneside]{memoir}
\usepackage[utf8]{inputenc}
\usepackage[T1]{fontenc}
\usepackage[spanish, es-tabla]{babel}
\addto\captionsspanish{
  \renewcommand{\listtablename}{Índice de Tablas}
}
\usepackage{amsmath,amssymb}
\usepackage{graphicx}
\usepackage{microtype}
\usepackage{booktabs}
\usepackage{physics}
\usepackage{nameref}
\usepackage{eso-pic}
\usepackage{float}
\usepackage{caption}
\usepackage{csquotes} 
\usepackage[absolute,overlay]{textpos}
\usepackage{array}
\usepackage{multirow}
\usepackage{tabulary}
\usepackage{subcaption}
\usepackage{tikz}
\usepackage[backend=biber,style=iso-numeric,sorting=nyt]{biblatex}
\DeclareLanguageMapping{spanish}{spanish-apa}
\addbibresource{bibliografia.bib}
\usepackage[dvipsnames,svgnames]{xcolor}
\usepackage{imakeidx}
\usepackage{hyperref}
\hypersetup{
colorlinks,
citecolor=MidnightBlue,    % Muy oscuro
linkcolor=DarkBlue,        % Oscuro
urlcolor=DarkBlue,        % Medio
filecolor=DarkSlateBlue    % Oscuro
}
\makeindex[
    title=Índice Alfabético,
    columns=2,                          % 2 columnas
    columnsep=1cm,                  % Espacio entre columnas
    intoc]                              % Que aparezca en el índice general


\usepackage{titlesec}
\usepackage{acronym}

\definecolor{chaptercolor}{RGB}{0,0,0}
\titleformat{\chapter}[display]
{\normalfont\bfseries\color{chaptercolor}}
{\raggedright\Huge\MakeUppercase{\chaptertitlename}\ {\fontsize{40}{45}\selectfont\thechapter}} % Capítulo X
{20pt}
{\titlerule\vspace{10pt}\Huge\raggedleft} % Línea + Título
[\vspace{10pt}\titlerule] % Línea final
\titlespacing*{\chapter}{0pt}{50pt}{30pt}

\addto\captionsspanish{\renewcommand{\tablename}{Tabla}}

\captionnamefont{\normalfont\bfseries}    % "Tabla" en negrita
\captionstyle{\centering}                 % Centrado

\linespread{1.5}


\title{Estudio Hidrodinámico de la Interacción entre Eyecciones de Masa Coronal}
\author {Cristian David Chavez Aponte}
\date{\today}
\newcommand{\tutor}{Dr. Miguel Andrés Paez Murcia}
\newcommand{\universidad}{Universidad Pedagógica y Tecnológica de Colombia}
\newcommand{\facultad}{Facultad de Ciencias Básicas, Escuela de Física}
\newcommand{\programa}{Escuela de Física}
\newcommand{\lugar}{Tunja, Boyacá, Colombia}


\setlrmarginsandblock{3.5cm}{2.5cm}{*}
\setulmarginsandblock{2.5cm}{2.5cm}{*}
\checkandfixthelayout

\usepackage{transparent} % Permite control de opacidad con \transparent

% Marca de agua con opacidad
\newcommand\BackgroundPic{
  \put(0,0){
    \parbox[b][\paperheight]{\paperwidth}{
      \vfill
      \centering
      \transparent{0.05}
      \includegraphics[width=0.7\paperwidth]{Imágenes/escudo.png}
      \vfill
    }
  }
}
\newcommand{\margenIzquierdo}{
  \AddToShipoutPictureBG*{
    \ifnum\value{page}>2
      \begin{tikzpicture}[remember picture,overlay]
        \draw[white, thick, line width=2pt] (current page.south west) ++(2.55cm,5) -- ++ (0,25);
      \end{tikzpicture}
    \fi
  }
} 
\AtBeginShipout{\margenIzquierdo}
